\documentclass[12pt]{article}
\usepackage{amsmath}
\usepackage{amssymb}
\usepackage{geometry}
\usepackage{fancyhdr}
\geometry{a4paper, margin=1in}
\pagestyle{fancy}
\fancyhf{}
\fancyhead[L]{\textbf{Mathraining}}
\fancyhead[R]{\small{9 juillet 2024}}

\begin{document}

\section*{Problème \#4227}

La suite $(a_i)_{i \in \mathbb{N}}$ satisfait $a_{m+n} + a_{m-n} = \frac{1}{2}(a_{2m}+a_{2n})$ pour tous $m, n \in \mathbb{N}$ avec $m \geq n$. 
Si $a_1 = 1$, déterminer les valeurs que peut prendre $a_{1995}$. \\

On change la formule en : 
$$2a_{m+n} + 2a_{m-n} = a_{2m} + a_ {2n}$$

Pour $n=m$ : $$2a_{m} + 2a_{m} = a_{2m}  \iff \boxed{a_{2m} = 4a_{m}}$$

Pour $n=0$, $n=1$ et $n=2$ :
$$2a_{2m} + 2a_{0} = a_{2m} + a_{2m} \iff \boxed{a_0 = 0}$$
$$2a_{m+1} + 2a_{1} = a_{2m} + a_{2} \iff 2a_{m+1} + 2 = 4a_{m} + 4a_1 \iff \boxed{a_{m+1} = 2a_{m} + 2}$$
$$2a_{m+2} + 2a_{2} = a_{2m} + a_{4} \iff 2a_{m+2} + 8 = 4a_{m} + 4a_2 \iff \boxed{a_{m+2} = 2a_{m} + 8}$$
$$2a_{m+3} + 2a_{3} = a_{2m} + 2a_{3} \iff 2a_{m+3} + 8 = 4a_{m} + 4a_2 \iff \boxed{a_{m+2} = 2a_{m} + 8}$$
$$\cdots$$

On trouve comme forme générale respectivement : 
$$a_n = 3\cdot2^{n-1}-2 \text{ pour tout } n \ge 1$$
\end{document}
