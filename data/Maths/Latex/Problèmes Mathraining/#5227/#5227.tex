\documentclass[12pt]{article}
\usepackage{amsmath}
\usepackage{amssymb}
\usepackage{geometry}
\usepackage{fancyhdr}
\setlength{\parindent}{0pt}
\geometry{a4paper, margin=1in}
\pagestyle{fancy}
\fancyhf{}
\fancyhead[L]{\textbf{Mathraining}}
\fancyhead[R]{\small{16 juillet 2024}}

\begin{document}

\section*{Problème \#5327}
Soit $f : \mathbb{Z} \to \mathbb{Z}$ une fonction telle que
$$f(f(x)-y) = f(y) - f(f(x)) \ \text{ pour tous } x, y \in \mathbb{Z}.$$
Montrer que $f$ est bornée, c'est-à-dire qu'il existe $C > 0$ tel que $-C \leq f(x) \leq C$ pour tout $x \in \mathbb{Z}.$
\par\vspace{-.5\ht\strutbox}\noindent\hrulefill\par

Supposons $k$ point fixe de $f:$
$$f(k)=k \iff f(f(k)) = f(k) \iff f(k) = 0 \iff k=0$$
$$\boxed{f(k) = k \iff k=0}$$

En posant $y=f(x):$
$$\boxed{f(0) = 0}$$

Puis en posant $y=0:$
$$f(f(x)) = -f(f(x)) \iff \boxed{f(f(x)) = 0}$$

Donc $:$
$$\boxed{f(f(x) - y) = f(y)}$$

Avec $x=0$ on obtient :
$$\boxed{f(-y) = f(y)}$$
D'où $f$ paire. \\

Donc en général $:$
$$f(f(x) + y) = f(y), \forall x,y \in \mathbb{Z}$$
Ou bien 
$$f(f(x) - f(y)) = 0 , \forall x,y \in \mathbb{Z}$$
\end{document}
